\documentclass{article}[12 pt]

\title{Diseñando la ICT de un chalet}
\author{Carlos Ortega Marchamalo \& Pablo Collado Soto}
\date{}

\begin{document}
	\maketitle

	\section{Introducción}
		Antes de inventar una vivienda para posteriormente diseñar su Infraestructura Común de Telecomunicación o de escoger una ICT ya diseñada para analizarla hemos decidido estudiar una instalación real. La ICT a la que nos atendremos durante el informe pertenece a una vivienda unifamiliar de la Urbanización Sotolargo en la Provincia de Guadalajara. Hemos optado por estudiar la ICT de un chalet en vez de la de un bloque de edificios para profundizar en este tipo de inmueble ya que es el que menos hemos estudiado en las clases teóricas de la asignatura. Asimismo, ésto nos permite acercarnos a una situación real que en ocasiones queda muy lejos de las aulas. Con ello esperamos ser capaces de tener en cuenta las limitaciones que nos impiden aplicar directamente la teoría al mundo real.\\

		Además de estudiar la \textbf{ICT} propiamente dicha hemos decidido también abordar la red telefónica y de datos. Al situarse el inmueble en una zona alejada de grandes ciudades veremos que el acceso es todavía a través de una línea ADSL (\textbf{A}bonado \textbf{Digital} \textbf{A}simétrico) por lo que lo compararemos con los accesos actuales basados en su mayoría en fibra óptica.\\

		Las frecuencias analiadas se encuentran en la banda \textbf{UHF} (\textbf{U}ltra \textbf{H}igh \textbf{F}requencies) ya que nos centraremos en la infraestructura de \textbf{T}elevisión \textbf{D}igital \textbf{T}errestre. A fin de no alargar el informe de manera innecesaria señalamos que todas las atenuaciones y características de los equipos analiados se ciñen a esta banda.\\

	\section{Fuentes de información}
		Con la intención de ajustarnos en la medida de lo posible a la realidad hemos consultado los planos de obra de la vivienda. No obstante hemos encontrado que, si bien se aludía a que la ICT vigente en el momento de la construcción era \textbf{XXXX}, no se incluían detalles sobre la misma. Tan solo se señalaba la localización de una serie de tomas de televisión y teléfono sin mostrar las canalizaciones hasta las mismas.\\

		Debido a esto, y tras consultar con el arquitecto de la vivienda nos decidimos a investigar la localización del registro primario para intentar reconstruir la ICT a partir de los equipos y líneas de transmisión instaladas. Asimismo, a la hora de calcular las atenuaciones en los cables hemos aproximado las distancias a partir de la planta acotada que se incluía en los planos de la vivienda. A pesar de que esto aporta una mayor fidelidad a los resultados debemos reconocer que no se ajustan de manera totalmente estricata a la realidad. Al observar los equipos instalados hemos visto que las líneas de transmisión discurren por el interior de los distintos tabiques, suelos y paredes con lo que deberemos tolerar esta pequeña inexactitud.

	\section{Equipamiento}
		Con tan solo retirar una cubierta de PVC hemos tenido acceso al registro primario del inmueble. Dentro hemos encontrado el siguiente equipo:

		\begin{tabular}{| c | c | c |}
			\hline
			\textbf{Equipo} & \textbf{Fabricante} \textbf{Referencia}\\
			\hline
			Fuente de alimentación con 2 salidas & Televés & 5495\\
			\hline
			PAU/Repartidor & Televés & 5436\\
			\hline
			Punto de Terminación de Red & Telefónica & No Disponible\\
			\hline
			Splitter & Telefónica & No Disponible\\
			\hline
			Router/Módem & Telefónica & No Disponible\\
			\hline
			Tomas TV-RF/SAT & Televés & 5226\\
			\hline
			Cable Coaxial T-100 Plus  & Televés & 2141\\
			\hline
			Cable Cobre CTX Cu & Televés & 2138\\
			\hline
		\end{tabular}

		Entre el equipo observado encontramos 2 elementos que \textit{a priori} nos resultan extraños. En primer lugar encontramos una fuente de alimentación. Si pensamos en la estructura de la ICT nos daremos cuenta de que la señal es captada por una antena diseñada para la banda de televisión y que esta deberá ser amplificada dada su aplitud tan reducida. Si modeláramos el circuito esta antena sería nuestro generador y la carga equivalente vendría a representar tanto las líneas de distribución como las cargas de los equipos conectados a las mismas.\\

		El hecho de que la señal de entrada a la \textbf{ICT} sea tan pequeña nos lleva a preguntarnos por qué no hemos incluido un amplificador en la tabla anterior. Destacamos pues que este amplificador de cabecera no se encuentra en el registro primario sino más cerca de la antena. Es por ello que no hemos tenido acceso al equipo en cuestión pero trás indagar llegamos a una configuración de equipos de Televés muy común en viviendas de este tipo que podemos ver en la figura \textbf{INSERTAR FIGURA Y REFERENCIA}. En ellas se incluye, como no podría ser de otra manera, un amplificador de cabecera con referencia \textbf{XXX}. Pasamos a hacer un pequeño inciso sobre el cálculo de ganancias y su uso a la hora de obtener las atenuaciones en la infraestructura analizada\\

		\subsection{Frecuencias, ganancias y $decibelios$}
			Los amplificadores son por definición circuitos activos (contienen elementos activos como transistores) ya que nos permiten manejar ganancias mayores que la unidad. Recordemos la definición de ganancia de un circuito:

			$$Sean v_i, v_o\ tensiones \rightarrow G = \frac{v_o}{v_i}$$

			Esto es, la ganancia es la relación entre las señales de entrada a un circuito y su salida. En este caso $v_i$ sería la señal de entrada al amplificador y $v_o$ la señal a la salida. De lo anterior se sigue que $G$ es un valor adimensional que caracteriza el amplificador utilizado. A pesar de lo aquí comentado debemos señalar que las señales $v_x$ son en general funciones del tiempo ($v(t)$) con lo que llevan asociado un espectro $V(w) = \mathcal{F}\{v(t)\}$ lo que supone una variación de esta ganancia con la frecuencia de las señales que manejos. Esto se resume estableciendo que $G \neq\ cte$ sino que es una función de la frecuencia $G(w)$.\\

			Dada esta variabilidad de la ganania los fabricantes acompañan sus productos de un ancho de banda de trabajo para sus aparatos. Esto es una banda de frecuencias para las que se puede asumir que $G =\ cte$. Estas ganancias son las que nosotros veremos en las hojas del fabricante, en este caso Televés, cuando analicemos la ICT.\\

			No obstante al acudir a las hojas de caracterísitcas de los equipos nos percataremos de que la unidad de esta ganancia son $dB$, es decir, decibelios. A pesar de toda la confusión que éstas generan (por lo menos a nosotros) las unidades logarítmicas están pensadas para facilitar el manejo de cantidades grandes. Dado que el $Belio$ es una unidad demasiado grande comunmente trabajaremos con el $decibelio$. Al igual que con otras unidades se cumple que $1\ Belio =\ 10\ decibelios$. Con todo, las ganancias que podemos esperar son del tipo:

			$$G(dB) = 10 \cdot log(\frac{V_o}{V_i})$$

			Nos damos cuenta en un prinipio de que los decibelios son en realidad adimensionales al igual que otras unidades como los $radianes$. Al final estamos comparando dos magnitudes. Además, dadas las propiedades de los logaritmos veremos que en vez de multiplicar por las ganancias podemos sumar tensiones y decibelios siempre que las tensiones también estén expresadas en unidades logarítmicas. Ésto se sigue de la definición misma de la ganancia que hemos visto anteriormente:

			$$G(dB) = 10 \cdot log(\frac{V_o}{1\ V}) - 10 \cdot log(\frac{V_i}{1\ V}) \rightarrow G(dB) = V_o(dBV) - V_i(dBV)$$

			Así llegamos a que $V_o(dBV) = G(dB) + V_i(dBV)$. Con este pequeño desarrollo explicamos el por qué de los cáculos que iremos haciendo a lo largo del informe. No debemos olvidar que podemos modelar cualquier línea de transmisión como los cables coaxiales de nuestra instalación como si se tratara de una "caja negra" con una relación de transmisión $\frac{V_o}{V_i}$ idéntica a la de este caso. Es por ello que si entendemos las atenuaciones como ganancias negativas el procedimiento para calcular todas las pérdidas de señal a lo largo de la instalación es análogo a éste. Esperamos haber disipado las dudas que podríamos tener sobre la forma de operar con estas undidades que pueden jugarnos una mala pasada si no tenemos cuidado...\\

		\subsection{¿Qué son esos equipos?}
			Si queríamos dejar algo claro es que elementos activos como los amplificadores necesitan un lugar del que sacar la potencia que "inyectan" en su salida. Esto es, necesitamos alimentarlos. Es aquí donde se encuadra la fuente de alimentación que incluíamos en la relación de equipos. Este aparato se encargará de suministrar la alimentación (una tensión contínua) requerida por el amplificador y por la antena. Además tal y como vemos en \textbf{INSERTAR FIGURA Y REFERENCIA} se va a encargar de recibir esta señal amplificada.\\

			Llegados a este punto nos encontramos con una gran diferencia respecto a las instalaciones en edificios residenciales de varias planata a las que estábamos acostumbrados. Esta fuente de alimentación se encarga de relegar esta señal recibida al \textbf{P}unto de \textbf{A}cceso de \textbf{U}suario que recogíamos antes \textbf{Y} de entreagar la señal directamente a una toma.\\

			Si nos adelantamos un poco a los acontecimientos podemos señalar que la \textbf{ICT} estudiada cuenta con $4$ tomas de señal. Si nos fijamos fijamente en el \textbf{PAU/Repartidor} de la figura \textbf{INSERTAR REFERENCIA} nos percatamos de que tiene una relación de entrada salida $1:3$, esto es, lleva la señal de entrada a las $3$ tomas restantes. Es por esto que sabemos que la fuente de alimentación debe entregar directamente a la toma restante.\\

			Como todo equipo emplear una fuente de alimentación no es "gratis" en el sentido de que conllevará una serie de pérdidas que tendremos que contabilizar y que afectarán a todas las tomas del inmueble.\\

		\subsection{Red Telefónica y de Datos}
			Ya en la introducción

\end{document}
