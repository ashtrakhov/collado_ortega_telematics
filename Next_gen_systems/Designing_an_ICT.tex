\documentclass{article}[12 pt]

\title{Diseñando la ICT de un chalet}
\author{Carlos Ortega Marchamalo \& Pablo Collado Soto}
\date{}

\begin{document}
	\maketitle

	\section{Introducción}
		Antes de inventar una vivienda para posteriormente diseñar su Infraestructura Común de Telecomunicación o de escoger una ICT ya diseñada para analizarla hemos decidido estudiar una instalación real. La ICT a la que nos atendremos durante el informe pertenece a una vivienda unifamiliar de la Urbanización Sotolargo en la Provincia de Guadalajara. Hemos optado por estudiar la ICT de un chalet en vez de la de un bloque de edificios para profundizar en este tipo de inmueble ya que es el que menos hemos estudiado en las clases teóricas de la asignatura.
	\section{Fuentes de información}
		Con la intención de ajustarnos en la medida de lo posible a la realidad hemos consultado los planos de obra de la vivienda. No obstante hemos encontrado que, si bien se aludía a que la ICT vigente en el momento de la construcción era \textbf{XXXX}, no se incluían detalles sobre la misma. Tan solo se señalaba la localización de una serie de tomas de televisión y teléfono sin mostrar las canalizaciones hasta las mismas.
		\\
		% Consultar el nombre del registro primario en chalets!
		Debido a esto, y tras consultar con el arquitecto de la vivienda nos decidimos a investigar la localización del registro primario para intentar reconstruir la ICT a partir de los equipos y líneas de transmisión instaladas. Asimismo, a la hora de calcular las atenuaciones en los cables hemos aproximado las distancias a partir de la planta acotada que se incluía en los planos de la vivienda. A pesar de que esto aporta una mayor fidelidad a los resultados debemos reconocer que no se ajustan de manera totalmente estricata a la realidad. Al observar los equipos instalados hemos visto que las líneas de transmisión discurren por el interior de los distintos tabiques y paredes con lo que deberemos tolerar esta pequeña inexactitud.
	\section{Equipamiento}
		Con tan solo retirar una cubierta de PVC hemos tenido acceso al registro primario del inmueble. Dentro hemos encontrado el siguiente equipo:

		\begin{tabular}{| c | c |}
			\hline
			\textbf{EQUIPO} & \textbf{REFERENCIA}\\
			\hline
			Fuente de alimentación con 2 salidas & Televés 5495\\
			\hline
			PAU/Repartidor & Televés 5436\\
			\hline
			Punto de Terminación de Red & Telefónica\\
			\hline
			Splitter & Telefónica\\
			\hline
			Router/Módem & Telefónica\\
			\hline
			Tomas TV-RF/SAT & Televés 5226\\
			\hline
			Cable Coaxial T-100 Plus  & Televés 2141\\
			\hline
			Cable Cobre CTX Cu & Televés 2138\\
			\hline
		\end{tabular}


\end{document}
