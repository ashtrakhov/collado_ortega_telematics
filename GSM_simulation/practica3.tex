\documentclass[12pt]{article}

\usepackage[utf8]{inputenc}

\usepackage{geometry}
\geometry{
 a4paper,
 left = 19mm,
 right = 19mm,
 top = 15mm,
 }

\newcommand{\newpar} {
	\vskip 1cm
}

\title{\textbf{Simulación de una red GSM}}
\author{Carlos Ortega Marchamalo \& Pablo Collado Soto \\ \\ \textbf{Conmutación} \\ \textit{Universidad de Alcalá}}
\date{}

\begin{document}
	\maketitle

	\newpage
	\tableofcontents
	\newpage

	\section{Introducción}
		El paso del tiempo ha desencadenado cambios en muchos aspectos y ámbitos y las redes de comunicaciones no iban a ser menos. En un principio estas eran muy simples y rudimentarias y los avances tecnológicos introdujeron modificaciones en ellas hasta llegar al punto de nuestro interés y estudio: \textit{las redes GSM}.

		La simulación de estos sistemas, tal y como ya hemos podido constatar, nos brinda nuevas posibilidades y nos permite reproducir el escenario que deseemos para corroborar la precisión y exactitud de los cálculos teóricos realizados previamente a cerca de él así como apreciar posibles variaciones entre ambos procedimientos. Con este objetivo en mente partiremos de las nociones y conocimientos adquiridos en la parte teórica de la asignatura para, a partir de ellos, abordar los ya mencionados cálculos teóricos y posteriormente implementar el modelo en el simulador, analizando los resultados obtenidos a través de él y comparándolos con los extraídos con anterioridad.

		Así, en primer lugar llevaremos a cabo una breve explicación del sistema del que disponemos en su conjunto para proseguir con su análisis matemático, centrándonos, por un lado, en las llamadas y el bloqueo que estas sufren debido a las condiciones y circunstancias de la red existente y, por otra parte, examinando qué sucede en cuanto a los SMS y los retardos que padecen en función de las capacidades de los enlaces por los cuales transitan. Concluiremos plasmando todo esto en el COMNET, teniendo que establecer ciertos ajustes para poder asimilar el esquema con la realidad a causa de ciertas carencias que este programa presenta.

		Esperemos haber sido capaces de plasmar esta información de una forma clara y condisa para así comprender de un mejor modo su contenido sin haber alcanzado el punto de la reiteración o complejidad en las descripciones elaboradas.

	\section{Descripción de la red empleada}
\end{document}